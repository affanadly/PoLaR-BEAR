Gelombang radio sejenak atau \textit{Fast radio bursts} (FRB) ialah pancaran radio jalur lebar yang terang dengan jangka masa milisaat atau kurang. Fenomena astrofizik misteri ini diclasifikasikan oleh kelewatan isyarat yang mana ia bergantung pada frekuensi, juga dikenali sebagai ukuran penyebaran atau \textit{dispersion measure} (DM). Hampir kesemua FRB yang telah dikesan mempunyai DM yang mana penyebarannya melebihi daripada apa yang disumbangkan oleh Bima Sakti. Ini menjadi bukti bahawa FRB berasal dari luar Bima Sakti. Punca sebenar FRB masih tidak dapat diketahui, tetapi kebanyakan petanda menunjukkan FRB mungkin berasal dari \textit{magnetar} iaitu bintang neutron yang mempunyai medan magnet yang sangat kuat. Bagi mengesahkan dan memahami asal usul dan mekanisme emisi FRB, lebih banyak sampel FRB diperlukan. Dalam kajian ini, sebuah program untuk mengesan FRB daripada data teleskop radio telah dihasilkan menggunakan \textit{Python} berdasarkan program BEAR. Antara fungsi penting yang dijalankan dalam program ini ialah pebatan RFI atau \textit{RFI mitigation}, penyah-sebaran atau \textit{dedispersion}, dan penurasan padan atau \textit{matched filtering} berdasarkan ujian nisbah statistik kemungkinan. Program tersebut menghasilkan calon-calon FRB yang dikesan terdapat dalam data. Program tersebut berjaya mengesan semua FRB daripada data FRB sebenar dan data FRB yang terjana dengan sisihan setanding BEAR. Program ini diharap dapat memberikan pemahaman yang lebih mudah dalam metodologi pengesanan FRB.