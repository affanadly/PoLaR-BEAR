Fast radio bursts (FRB) are bright, broadband radio emissions with durations of milliseconds or less. These mysterious astrophysical phenomena are characterized by their frequency-dependent delay in the signal, also known as dispersion measure or DM. Almost all FRBs detected have DMs more than the contribution from the Milky Way, suggesting that they are extragalactic. The exact source of FRBs is uncertain, but most evidence seem to point towards magnetars, which are neutron stars with strong magnetic fields. Finding a larger sample of FRBs would be paramount in confirming and understanding the exact progenitor and burst mechanism for FRBs. Here, a Python program to detect FRBs in radio telescope data is developed based on BEAR. The program performs various functions, most importantly RFI mitigation, dedispersion, and matched filtering based on the likelihood statistic ratio test. The program then outputs FRB candidates which are detected in the data. The program manages to detect all FRBs in the real FRB data and generated FRB data with deviations comparable to BEAR. The program is expected to provide a simpler understanding in the FRB detection methodology.\\

\noindent\textbf{Key words:} astrophysics -- transients: fast radio bursts -- methods: data analysis