\chapter{Summary}\label{summary}

FRBs are bright, broadband radio emission ranging from milliseconds or less. The key characteristic of FRBs is their dispersion measure or DM, seen as a frequency-dependent delay in the signal. Most FRBs have DMs more than the contribution from our galaxy, suggesting its extragalactic nature. The signal characteristics seem to suggest that FRBs come from magnetars. FRBs are usually detected using single-dish telescopes, and the data is passed through pipelines that perform the detection and analysis of the FRBs. 

% As stated earlier, the objectives of this project are to understand FRB detection methods, which is mainly based on the likelihood statistic ratio, to develop a program that independently searches for FRBs in radio telescope data, for which I developed PoLaR BEAR, to compare its performance with BEAR, where its performance can be seen to the comparable, and to test the performance increase for GPU dedispersion, for which \texttt{CuPy} can be seen to provide up to 8 times speedup for dedispersion.

% Here, I created the PoLaR BEAR independent FRB detection program that searches of FRBs in radio telescope data in Python, based on BEAR. The program involves many functions, where the three main ones are RFI mitigation, dedispersion, and matched filtering. RFI mitigation is performed by zapping frequency channels with RFI, and by using ZDMF to remove any narrow-band short-duration RFI with no dispersive nature. Dedispersion shifts the data to compensate for the dispersion to maximize the SNR of the FRB. Matched filtering then calculates the SNR in the data based on the likelihood statistic ratio for all DM, time, and W, then peaks in the SNR denotes an FRB detection. 

% PoLaR BEAR is tested with real FRB data and fake FRB data generated using Python. PoLaR BEAR performs well with small deviations in the detection parameters compared to the actual parameters, comparable to that of BEAR. The detection SNR are also slightly underestimated as expected due to deviations in DM and W. GPU dedispersion was also tested in PoLaR BEAR using \texttt{CuPy}, and the process can be sped up by up to 8 times compared to brute force dedispersion. 

% With this, an independent FRB detection pipeline was successfully created using Python. Detecting more FRBs will help in pinpointing the correct origin of FRBs. If they are from magnetars, FRBs would be a crucial way to study their properties e.g. the origin of their strong magnetic fields, their composition, their formation origin, and the types of stars that would transition into them.

The first objective of this project was to understand FRB detection methods, which are mainly done using matched filtering and the likelihood statistic ratio. 

The second objective was to develop a program that independently searches for FRBs in radio telescope data. Here, I created the PoLaR BEAR independent FRB detection program that searches of FRBs in radio telescope data in Python, based on BEAR. The program involves many functions, where the three main ones are RFI mitigation, dedispersion, and matched filtering. RFI mitigation is performed by zapping frequency channels with RFI, and by using ZDMF to remove any narrow-band short-duration RFI with no dispersive nature. Dedispersion shifts the data to compensate for the dispersion to maximize the SNR of the FRB. Matched filtering then calculates the SNR in the data based on the likelihood statistic ratio for all DM, time, and W, then peaks in the SNR denotes an FRB detection. 

The third objective was to compare the performance of the program developed with BEAR. PoLaR BEAR was tested with real FRB data and fake FRB data generated using Python. PoLaR BEAR performs well with small deviations in the detection parameters compared to the actual parameters, comparable to that of BEAR. The detection SNR are also slightly underestimated as expected due to deviations in DM and W.

The final objective was to test the performance increase for GPU dedispersion. In this case, GPU dedispersion was implemented in PoLar BEAR using \texttt{CuPy}, and dedispersion can be sped up by up to 8 times compared to brute force CPU dedispersion.

With this, an independent FRB detection pipeline was successfully created using Python. Detecting more FRBs will help in pinpointing the correct origin of FRBs. If they are from magnetars, FRBs would be a crucial way to study their properties e.g. the origin of their strong magnetic fields, their composition, their formation origin, and the types of stars that would transition into them.